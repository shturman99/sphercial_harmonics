\documentclass[11pt]{article}

% =========================
% Basic packages
% =========================
\usepackage[a4paper,margin=1in]{geometry}
\usepackage{amsmath,amssymb,amsfonts}
\usepackage{bm}
\usepackage{physics}
\usepackage{mathtools}
\usepackage{hyperref}
\usepackage{enumitem}

% =========================
% Notation & convenience
% =========================
\newcommand{\Ylm}{Y_{\ell m}}
\newcommand{\Ylmp}{Y_{\ell' m'}}
\newcommand{\CG}[6]{\langle #1\,#2\,#3\,#4 \mid #5\,#6 \rangle}
\newcommand{\ThreeJ}[6]{\begin{pmatrix} #1 & #2 & #3 \\ #4 & #5 & #6 \end{pmatrix}}

% =========================
% Derivation boxes
% =========================
\newenvironment{derivationbox}{
  \begin{center}
  \begin{minipage}{0.95\linewidth}
  \hrule\vspace{0.5em}
  \textbf{Derivation}\par
  \vspace{0.5em}
}{
  \vspace{0.5em}
  \hrule
  \end{minipage}
  \end{center}
}

% =========================
% Document metadata
% =========================
\title{Scratch Notes: Bipolar Spherical Harmonics for Gravitational Waves}
\author{Murman Gurgenidze}

\begin{document}
\maketitle

\tableofcontents
\newpage

% ============================================================
\section{Conventions and definitions}
\begin{tabular}{p{0.6\textwidth} p{0.3\textwidth}}
\textbf{Definition} & \textbf{Concept} \\ \hline
$\nabla^2\phi = 0$ 
& Laplace equation \\[0.5em]
\hline
$\nabla^2 \phi = - k^2 \phi$ \\Is elliptic partial differential equation, which arises as an eigenvalue problem of the Laplace operator.
& Helmholtz equation \\
\end{tabular}

% ============================================================
\section{Global Conventions and Notation}
% ============================================================

\subsection{Spherical Harmonics Normalization}
We take $\hat n$ as a unit direction on $S^2$ and define $Y_{\ell m}(\hat n)$ so that
\begin{equation}
  \int_{S^2} d\hat n\; Y_{\ell m}(\hat n) Y^*_{\ell' m'}(\hat n)
  =
  \delta_{\ell\ell'}\delta_{mm'} .
\end{equation}
This normalization follows standard angular-momentum conventions \cite{EdmondsPlaceholder,VarshalovichPlaceholder}.

\subsection{Integration Measure on $S^2$}
We use
\begin{equation}
  d\hat n \equiv d\Omega = \sin\theta\, d\theta\, d\phi, \qquad
  \int_{S^2} d\hat n = 4\pi .
\end{equation}

\subsection{Phase Convention}
We adopt the Condon--Shortley phase, so that
\begin{equation}
  Y_{\ell,-m}(\hat n) = (-1)^m Y_{\ell m}^*(\hat n) .
\end{equation}

\subsection{CG / Wigner 3j Relation}
Clebsch--Gordan coefficients are related to Wigner 3j symbols by
\begin{equation}
  \CG{\ell_1}{m_1}{\ell_2}{m_2}{L}{M}
  =
  (-1)^{\ell_1-\ell_2+M}
  \sqrt{2L+1}
  \ThreeJ{\ell_1}{\ell_2}{L}{m_1}{m_2}{-M} .
\end{equation}

\subsection{Spin-Weight Convention}
Spin-weighted harmonics follow the Goldberg \emph{et al.} normalization:
${}_sY_{\ell m}(\hat n)$ are generated by the spin-raising/lowering operators
$\eth$ and $\bar\eth$ acting on $\Ylm$ with the standard phase choice
\cite{GoldbergPlaceholder}. We use the complex-conjugation rule
\begin{equation}
  {}_sY_{\ell m}^*(\hat n)
  =
  (-1)^{m+s}\; {}_{-s}Y_{\ell,-m}(\hat n) .
\end{equation}

\subsection{Complex Conjugation Rules}
For ordinary harmonics,
\begin{equation}
  Y_{\ell m}^*(\hat n) = (-1)^m Y_{\ell,-m}(\hat n) .
\end{equation}

\subsection{Parity Rules}
Under spatial inversion $\hat n \mapsto -\hat n$,
\begin{equation}
  Y_{\ell m}(-\hat n) = (-1)^{\ell} Y_{\ell m}(\hat n) .
\end{equation}

\subsection{Sanity Checks}
% These checks are included to catch sign/phase slips and normalization mistakes early.
% They prevent silent convention drift when combining CG coefficients, parity rules,
% and complex conjugation identities in later sections.
\begin{align}
  \int_{S^2} d\hat n\; Y_{\ell m}(\hat n) Y^*_{\ell' m'}(\hat n)
  &= \delta_{\ell\ell'}\delta_{mm'} ,
  \label{eq:sanity-orthonormality} \\
  Y_{\ell m}^*(\hat n) &= (-1)^m Y_{\ell,-m}(\hat n) ,
  \label{eq:sanity-conjugation} \\
  Y_{\ell m}(-\hat n) &= (-1)^{\ell} Y_{\ell m}(\hat n) ,
  \label{eq:sanity-parity} \\
  \CG{\ell}{m}{\ell}{-m}{0}{0} &= (-1)^{\ell-m}\frac{1}{\sqrt{2\ell+1}} .
  \label{eq:sanity-cg}
\end{align}
These identities are standard and provide quick checks against phase/sign errors
in later BipoSH manipulations \cite{EdmondsPlaceholder,VarshalovichPlaceholder,WaldPlaceholder,MaggiorePlaceholder}.

\subsection{Metric and Signature}

\newpage

% ============================================================
\section{SO(3) and Angular Momentum Basics}
% ============================================================

\subsection{Angular Momentum Algebra}
\begin{align}
  [L_i, L_j] &= i \epsilon_{ijk} L_k \\
  L^2 \ket{\ell m} &= \ell(\ell+1) \ket{\ell m} \\
  L_z \ket{\ell m} &= m \ket{\ell m}
\end{align}

\subsection{Ladder Operators}
\begin{equation}
  L_\pm \ket{\ell m}
  =
  \sqrt{\ell(\ell+1)-m(m\pm1)}\ket{\ell,m\pm1}
\end{equation}

\newpage

% ============================================================
\section{Ordinary Spherical Harmonics}
% ============================================================

\subsection{Definition}
\begin{equation}
  Y_{\ell m}(\theta,\phi)
  =
  (-1)^m
  \sqrt{\frac{2\ell+1}{4\pi}\frac{(\ell-m)!}{(\ell+m)!}}
  P_\ell^m(\cos\theta)e^{im\phi}
\end{equation}

\subsection{Orthogonality}
\begin{equation}
  \int d\hat n\;
  Y_{\ell m}(\hat n) Y^*_{\ell' m'}(\hat n)
  =
  \delta_{\ell\ell'}\delta_{mm'}
\end{equation}

\subsection{Completeness}
\begin{derivationbox}
Completeness on $S^2$ follows from the addition theorem:
\begin{align}
  \sum_{m=-\ell}^{\ell}
  Y_{\ell m}(\hat n) Y^*_{\ell m}(\hat n')
  &= \frac{2\ell+1}{4\pi} P_\ell(\hat n\cdot\hat n'), \\
  \sum_{\ell=0}^{\infty}\sum_{m=-\ell}^{\ell}
  Y_{\ell m}(\hat n) Y^*_{\ell m}(\hat n')
  &= \delta(\hat n,\hat n').
\end{align}
The second line is the completeness relation on the sphere, with the delta
function normalized so that $\int d\hat n\,\delta(\hat n,\hat n')=1$.
\end{derivationbox}

\newpage

% ============================================================
\section{Clebsch--Gordan Coefficients}
% ============================================================

\subsection{Tensor Product Decomposition}
\begin{equation}
  \mathcal{H}_{\ell_1} \otimes \mathcal{H}_{\ell_2}
  =
  \bigoplus_{L=|\ell_1-\ell_2|}^{\ell_1+\ell_2} \mathcal{H}_L
\end{equation}

\subsection{Selection Rules}
\begin{itemize}
  \item $m_1 + m_2 = M$
  \item $|\ell_1-\ell_2| \le L \le \ell_1+\ell_2$
\end{itemize}

\newpage

% ============================================================
\section{BipoSH from Representation Theory}
% ============================================================

\subsection{SO(3) Irreps and Tensor Products}
\begin{derivationbox}
An SO(3) rotation $R$ acts on the irrep basis by
\begin{equation}
  \hat U(R)\ket{\ell m} = \sum_{m'} D^{\ell}_{m' m}(R)\ket{\ell m'}.
\end{equation}
On the tensor product $\mathcal{H}_{\ell_1}\otimes\mathcal{H}_{\ell_2}$ the action is
\begin{align}
  \hat U(R)\ket{\ell_1 m_1}\ket{\ell_2 m_2}
  &= \sum_{m_1' m_2'} D^{\ell_1}_{m_1' m_1}(R)
  D^{\ell_2}_{m_2' m_2}(R)
  \ket{\ell_1 m_1'}\ket{\ell_2 m_2'}.
\end{align}
Thus the product representation decomposes into irreps labeled by $L$.
\end{derivationbox}

\subsection{Coupled Basis Construction}
\begin{derivationbox}
Define the coupled basis $\ket{\ell_1 \ell_2; L M}$ by Clebsch--Gordan coefficients:
\begin{equation}
  \ket{\ell_1 \ell_2; L M}
  = \sum_{m_1 m_2}
  \CG{\ell_1}{m_1}{\ell_2}{m_2}{L}{M}
  \ket{\ell_1 m_1}\ket{\ell_2 m_2}.
\end{equation}
Orthonormality of the CG coefficients implies
\begin{equation}
  \braket{\ell_1 \ell_2; L M | \ell_1 \ell_2; L' M'} = \delta_{L L'}\delta_{M M'}.
\end{equation}
This basis furnishes the irrep $\mathcal{H}_L$ inside
$\mathcal{H}_{\ell_1}\otimes\mathcal{H}_{\ell_2}$. \hfill [Edmonds], [Varshalovich]
\end{derivationbox}

\subsection{BipoSH from Representation Theory}
\begin{derivationbox}
Let $\ket{\hat n}$ be the position eigenstate on $S^2$, with
$Y_{\ell m}(\hat n)=\braket{\hat n|\ell m}$. The tensor-product position basis is
$\ket{\hat n_1,\hat n_2}=\ket{\hat n_1}\otimes\ket{\hat n_2}$. Then
\begin{align}
  Y^{LM}_{\ell_1\ell_2}(\hat n_1,\hat n_2)
  &\equiv \braket{\hat n_1,\hat n_2|\ell_1 \ell_2; L M} \\
  &= \sum_{m_1 m_2}
  \CG{\ell_1}{m_1}{\ell_2}{m_2}{L}{M}
  Y_{\ell_1 m_1}(\hat n_1)
  Y_{\ell_2 m_2}(\hat n_2),
\end{align}
which is the bipolar spherical harmonic definition derived from the coupled
basis. The labels $\ell_1,\ell_2$ describe the individual angular momenta on
$\hat n_1$ and $\hat n_2$, while $L,M$ label the total angular momentum.
\end{derivationbox}

Interpretation: why L labels anisotropy.
$L=0$ corresponds to a scalar under rotations of the pair $(\hat n_1,\hat n_2)$,
so any $L>0$ component measures deviations from statistical isotropy.

\newpage

% ============================================================
\section{Orthogonality and Completeness of BipoSH}
% ============================================================

\subsection{Orthogonality on $S^2\times S^2$}
\begin{derivationbox}
Using the definition and orthogonality of $Y_{\ell m}$,
\begin{align}
  &\int d\hat n_1 d\hat n_2\;
  Y^{LM}_{\ell_1\ell_2}(\hat n_1,\hat n_2)
  Y^{L'M'*}_{\ell_1'\ell_2'}(\hat n_1,\hat n_2) \\
  &= \sum_{m_1 m_2 m_1' m_2'}
  \CG{\ell_1}{m_1}{\ell_2}{m_2}{L}{M}
  \CG{\ell_1'}{m_1'}{\ell_2'}{m_2'}{L'}{M'}
  \int d\hat n_1\,Y_{\ell_1 m_1} Y^*_{\ell_1' m_1'}
  \int d\hat n_2\,Y_{\ell_2 m_2} Y^*_{\ell_2' m_2'} \\
  &= \delta_{\ell_1\ell_1'}\delta_{\ell_2\ell_2'}
  \sum_{m_1 m_2}
  \CG{\ell_1}{m_1}{\ell_2}{m_2}{L}{M}
  \CG{\ell_1}{m_1}{\ell_2}{m_2}{L'}{M'} \\
  &= \delta_{\ell_1\ell_1'}\delta_{\ell_2\ell_2'}\delta_{L L'}\delta_{M M'}.
\end{align}
The last step uses CG orthogonality.
\end{derivationbox}

\subsection{Completeness on $S^2\times S^2$}
\begin{derivationbox}
Completeness follows by expanding the product of delta functions:
\begin{align}
  \delta(\hat n_1,\hat n_1')\,\delta(\hat n_2,\hat n_2')
  &= \sum_{\ell_1 m_1} Y_{\ell_1 m_1}(\hat n_1) Y^*_{\ell_1 m_1}(\hat n_1')
  \sum_{\ell_2 m_2} Y_{\ell_2 m_2}(\hat n_2) Y^*_{\ell_2 m_2}(\hat n_2') \\
  &= \sum_{\ell_1\ell_2LM}
  Y^{LM}_{\ell_1\ell_2}(\hat n_1,\hat n_2)
  Y^{LM*}_{\ell_1\ell_2}(\hat n_1',\hat n_2'),
\end{align}
where the last line uses CG completeness to reorganize the sums over $m_1,m_2$.
\end{derivationbox}

\subsection{Phase-Convention Sensitivity}
\begin{derivationbox}
The orthogonality and completeness identities above rely on
\begin{equation}
  Y_{\ell,-m}(\hat n)=(-1)^m Y^*_{\ell m}(\hat n)
  \quad\text{and}\quad
  \sum_{m_1 m_2}
  \CG{\ell_1}{m_1}{\ell_2}{m_2}{L}{M}
  \CG{\ell_1}{m_1}{\ell_2}{m_2}{L'}{M'}
  = \delta_{L L'}\delta_{M M'}.
\end{equation}
If either phase convention is changed, the orthogonality of
$Y^{LM}_{\ell_1\ell_2}$ fails and the completeness relation no longer reproduces
$\delta(\hat n_1,\hat n_1')\delta(\hat n_2,\hat n_2')$.
\end{derivationbox}

\newpage

% ============================================================
\section{Isotropic Limit and Selection Rules}
% ============================================================

\subsection{$L=0$ BipoSH and Statistical Isotropy}
\begin{derivationbox}
For $L=0$ the CG coefficient enforces $\ell_1=\ell_2\equiv\ell$ and $m_2=-m_1$:
\begin{equation}
  \CG{\ell}{m}{\ell}{-m}{0}{0}
  = (-1)^{\ell-m}\frac{1}{\sqrt{2\ell+1}}.
\end{equation}
Therefore
\begin{align}
  Y^{00}_{\ell\ell}(\hat n_1,\hat n_2)
  &= \frac{(-1)^\ell}{\sqrt{2\ell+1}}
  \sum_m (-1)^m Y_{\ell m}(\hat n_1) Y_{\ell,-m}(\hat n_2) \\
  &= \frac{1}{\sqrt{2\ell+1}}
  \sum_m Y_{\ell m}(\hat n_1) Y^*_{\ell m}(\hat n_2),
\end{align}
which is proportional to the addition theorem and depends only on
$\hat n_1\cdot\hat n_2$.
\end{derivationbox}

\subsection{Recovery of the $C_\ell$ Expansion}
\begin{derivationbox}
For a statistically isotropic field, only $L=0$ contributes to the two-point
function:
\begin{align}
  \langle X(\hat n_1) X(\hat n_2) \rangle
  &= \sum_{\ell} C_{\ell}\, Y^{00}_{\ell\ell}(\hat n_1,\hat n_2) \\
  &= \sum_{\ell} \frac{C_{\ell}}{\sqrt{2\ell+1}}
  \sum_m Y_{\ell m}(\hat n_1) Y^*_{\ell m}(\hat n_2) \\
  &= \sum_{\ell} \frac{2\ell+1}{4\pi} C_{\ell} P_\ell(\hat n_1\cdot\hat n_2).
\end{align}
This recovers the usual isotropic $C_\ell$ expansion.
\end{derivationbox}

\subsection{Selection Rules}
\begin{derivationbox}
From the CG coefficients, BipoSH modes satisfy
\begin{align}
  m_1+m_2 &= M, \\
  |\ell_1-\ell_2| \le L \le \ell_1+\ell_2.
\end{align}
In the isotropic limit, $L=0$ additionally forces $\ell_1=\ell_2$.
These selection rules follow directly from SO(3) representation theory.
\end{derivationbox}


% ============================================================
\section{Spin-Weighted Spherical Harmonics}
% ============================================================

\subsection{Definition via $\eth$ and $\bar\eth$}
\begin{derivationbox}
Spin-weighted harmonics are generated from $Y_{\ell m}$ by applying the spin
raising and lowering operators $\eth$ and $\bar\eth$ with fixed normalization
\cite{GoldbergPlaceholder}:
\begin{align}
  {}_sY_{\ell m}
  &= \sqrt{\frac{(\ell-s)!}{(\ell+s)!}}\,\eth^s Y_{\ell m},
  && s \ge 0, \\[-0.25em]
  {}_sY_{\ell m}
  &= \sqrt{\frac{(\ell+s)!}{(\ell-s)!}}\,(-\bar\eth)^{-s} Y_{\ell m},
  && s \le 0.
\end{align}
These definitions fix the overall phase and normalization of
${}_sY_{\ell m}$ relative to $Y_{\ell m}$.
\end{derivationbox}

\begin{derivationbox}
The $\eth$ and $\bar\eth$ actions on spin-weighted harmonics are
\begin{align}
  \eth\, {}_sY_{\ell m} &= \sqrt{(\ell-s)(\ell+s+1)}\; {}_{s+1}Y_{\ell m}, \\[0.25em]
  \bar\eth\, {}_sY_{\ell m} &= -\sqrt{(\ell+s)(\ell-s+1)}\; {}_{s-1}Y_{\ell m}.
\end{align}
The square-root factors fix the normalization and are required for consistency
with the Condon--Shortley phase conventions.
\end{derivationbox}

\subsection{Normalization, Conjugation, and Parity}
\begin{derivationbox}
The orthonormality of the spin-weighted basis follows from the defining
relations above:
\begin{equation}
  \int d\hat n\; {}_sY_{\ell m}(\hat n)\,{}_sY^*_{\ell' m'}(\hat n)
  = \delta_{\ell\ell'}\delta_{mm'}.
\end{equation}
This identity fails if the $\eth/\bar\eth$ normalization or phase conventions
are altered.
\end{derivationbox}

\begin{derivationbox}
Complex conjugation and parity act as
\begin{align}
  {}_sY^*_{\ell m}(\hat n)
  &= (-1)^{m+s}\; {}_{-s}Y_{\ell,-m}(\hat n), \\[0.25em]
  {}_sY_{\ell m}(-\hat n)
  &= (-1)^{\ell}\; {}_{-s}Y_{\ell m}(\hat n).
\end{align}
Both identities are sensitive to the Condon--Shortley phase; any change of
phase propagates into parity and conjugation rules.\hfill [Edmonds], [Goldberg et al.]
\end{derivationbox}

\subsection{Specialization to $s=\pm2$}
\begin{derivationbox}
For gravitational waves the spin-2 harmonics are
\begin{align}
  {}_{+2}Y_{\ell m}
  &= \sqrt{\frac{(\ell-2)!}{(\ell+2)!}}\;\eth^2 Y_{\ell m}, \\[0.25em]
  {}_{-2}Y_{\ell m}
  &= \sqrt{\frac{(\ell+2)!}{(\ell-2)!}}\;\bar\eth^2 Y_{\ell m},
\end{align}
with the same normalization and conjugation rules as above.
\end{derivationbox}

\newpage

% ============================================================
\section{Spin-Weighted BipoSH}
% ============================================================

\subsection{Construction}
\begin{derivationbox}
For two fields with spin weights $s_1$ and $s_2$, define the spin-weighted
BipoSH basis by coupling their angular momenta:
\begin{equation}
  {}_{s_1 s_2}Y^{LM}_{\ell_1\ell_2}(\hat n_1,\hat n_2)
  = \sum_{m_1 m_2}
  \CG{\ell_1}{m_1}{\ell_2}{m_2}{L}{M}
  {}_{s_1}Y_{\ell_1 m_1}(\hat n_1)
  {}_{s_2}Y_{\ell_2 m_2}(\hat n_2).
\end{equation}
This reduces to the scalar BipoSH when $s_1=s_2=0$.
\end{derivationbox}

\subsection{Rotation Covariance}
\begin{derivationbox}
Spin-weighted harmonics transform under rotation $R$ as
\begin{equation}
  {}_sY_{\ell m}(R\hat n)
  = \sum_{m'} D^{\ell}_{m m'}(R)\; {}_sY_{\ell m'}(\hat n),
\end{equation}
so the coupled basis transforms as a single irrep:
\begin{equation}
  {}_{s_1 s_2}Y^{LM}_{\ell_1\ell_2}(R\hat n_1,R\hat n_2)
  = \sum_{M'} D^L_{M M'}(R)\; {}_{s_1 s_2}Y^{LM'}_{\ell_1\ell_2}(\hat n_1,\hat n_2).
\end{equation}
Thus $L,M$ label the total angular momentum of the pair.
\end{derivationbox}

\newpage

% ============================================================
\section{Gravitational-Wave Polarizations}
% ============================================================

\subsection{$+\!/\times$ and $R/L$ Bases}
\begin{derivationbox}
Define linear and circular polarization combinations by
\begin{align}
  h_R(\hat n) &= \frac{1}{\sqrt{2}}\big(h_+(\hat n) - i h_{\times}(\hat n)\big), \\[0.25em]
  h_L(\hat n) &= \frac{1}{\sqrt{2}}\big(h_+(\hat n) + i h_{\times}(\hat n)\big).
\end{align}
The inverse relations are
\begin{align}
  h_+(\hat n) &= \frac{1}{\sqrt{2}}\big(h_R(\hat n) + h_L(\hat n)\big), \\[0.25em]
  h_{\times}(\hat n) &= \frac{i}{\sqrt{2}}\big(h_L(\hat n) - h_R(\hat n)\big).
\end{align}
\end{derivationbox}

\subsection{Helicity and Spin Weight}
\begin{derivationbox}
The circular polarizations carry definite helicity and are expanded as
\begin{align}
  h_R(\hat n) &= \sum_{\ell m} h^R_{\ell m}\; {}_{-2}Y_{\ell m}(\hat n), \\[0.25em]
  h_L(\hat n) &= \sum_{\ell m} h^L_{\ell m}\; {}_{+2}Y_{\ell m}(\hat n).
\end{align}
Thus $s_R=-2$ and $s_L=+2$ track the helicity content explicitly.
\end{derivationbox}

\subsection{Parity and Chirality}
\begin{derivationbox}
Under spatial inversion $\hat n\mapsto-\hat n$,
\begin{align}
  h_+(\hat n) &\mapsto +h_+(-\hat n), \\[0.25em]
  h_{\times}(\hat n) &\mapsto -h_{\times}(-\hat n),
\end{align}
so parity swaps the circular modes:
\begin{equation}
  h_R(\hat n) \mapsto h_L(-\hat n),
  \qquad
  h_L(\hat n) \mapsto h_R(-\hat n).
\end{equation}
A chiral background corresponds to unequal statistics for $R$ and $L$.
\end{derivationbox}

\newpage

% ============================================================
\section{Gravitational-Wave Two-Point Functions}
% ============================================================

\subsection{Most General Correlator}
\begin{derivationbox}
For $A,A'\in\{R,L\}$ define the two-point function on $S^2\times S^2$ as
\begin{equation}
  \langle h_A(\hat n_1) h_{A'}(\hat n_2) \rangle
  = \sum_{\ell_1\ell_2LM}
  C^{AA'}_{\ell_1\ell_2L}
  {}_{s_A s_{A'}}Y^{LM}_{\ell_1\ell_2}(\hat n_1,\hat n_2),
\end{equation}
where $s_R=-2$ and $s_L=+2$. The coefficients $C^{AA'}_{\ell_1\ell_2L}$ encode
anisotropy ($L>0$) and chirality (differences between $RR$ and $LL$).
\end{derivationbox}

\subsection{Parity-Even and Parity-Odd Sectors}
\begin{derivationbox}
Under parity, the correlator satisfies
\begin{equation}
  \langle h_A(-\hat n_1) h_{A'}(-\hat n_2) \rangle
  = \sum_{\ell_1\ell_2LM} (-1)^{\ell_1+\ell_2}
  C^{AA'}_{\ell_1\ell_2L}
  {}_{-s_A,-s_{A'}}Y^{LM}_{\ell_1\ell_2}(\hat n_1,\hat n_2),
\end{equation}
while parity invariance requires this to equal
\begin{equation}
  \langle h_{\bar A}(\hat n_1) h_{\bar A'}(\hat n_2) \rangle,
  \qquad \bar R=L,\; \bar L=R.
\end{equation}
It is convenient to define parity-even/odd combinations
\begin{align}
  C^{AA'}_{\ell_1\ell_2L,\,\mathrm{even}}
  &= \tfrac{1}{2}\Big(C^{AA'}_{\ell_1\ell_2L}
  + (-1)^{\ell_1+\ell_2+L} C^{\bar A\bar A'}_{\ell_1\ell_2L}\Big), \\[0.25em]
  C^{AA'}_{\ell_1\ell_2L,\,\mathrm{odd}}
  &= \tfrac{1}{2}\Big(C^{AA'}_{\ell_1\ell_2L}
  - (-1)^{\ell_1+\ell_2+L} C^{\bar A\bar A'}_{\ell_1\ell_2L}\Big),
\end{align}
so that parity invariance enforces $C_{\mathrm{odd}}=0$.
\end{derivationbox}

\newpage

% ============================================================
\section{Sanity Checks \& Failure Modes}
% ============================================================

\begin{itemize}
  \item Phase-convention failures: using $Y_{\ell,-m}=(-1)^{m+1}Y^*_{\ell m}$ or
  a mismatched Condon--Shortley phase flips the sign of CG identities, breaking
  orthogonality and completeness.
  \item Parity sign mistakes: forgetting the $(-1)^\ell$ factor in
  $Y_{\ell m}(-\hat n)$ or the spin flip in
  ${}_sY_{\ell m}(-\hat n)=(-1)^\ell{}_{-s}Y_{\ell m}$ leads to incorrect
  parity-even/odd separation.
  \item Conjugation errors: dropping $(-1)^{m+s}$ in
  ${}_sY^*_{\ell m}=(-1)^{m+s}{}_{-s}Y_{\ell,-m}$ produces inconsistent
  $RR/LL$ correlator relations.
  \item CG normalization slips: any change in
  $\CG{\ell_1}{m_1}{\ell_2}{m_2}{L}{M}$ normalization spoils the coupled basis and
  invalidates the representation-theory derivation.
  \item Selection-rule oversights: ignoring $m_1+m_2=M$ or the triangle
  inequality leads to spurious BipoSH modes.
\end{itemize}

\end{document}
